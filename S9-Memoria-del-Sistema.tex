% Created 2025-01-13 Mon 11:58
% Intended LaTeX compiler: pdflatex
\documentclass[presentation]{beamer}
\usepackage[utf8]{inputenc}
\usepackage[T1]{fontenc}
\usepackage{graphicx}
\usepackage{longtable}
\usepackage{wrapfig}
\usepackage{rotating}
\usepackage[normalem]{ulem}
\usepackage{amsmath}
\usepackage{amssymb}
\usepackage{capt-of}
\usepackage{hyperref}
\nocite{*}
\usepackage[T1]{fontenc}
\usepackage[utf8]{inputenc}
\usepackage[spanish]{babel}
\usepackage[backend=biber, style=apa]{biblatex}
\addbibresource{/home/martincusme/repoArqui/bibliography.bib}
\usetheme{default}
\usecolortheme{}
\usefonttheme{}
\useinnertheme{}
\useoutertheme{}
\author{Lenin G. Falconí, Richard Dawkins, Jean LeCunn}
\date{}
\title{S9-Memoria-del-Sistema}

\hypersetup{
 pdfauthor={Lenin G. Falconí, Richard Dawkins, Jean LeCunn},
 pdftitle={S9-Memoria-del-Sistema},
 pdfkeywords={},
 pdfsubject={},
 pdfcreator={Emacs 27.1 (Org mode 9.3)}, 
 pdflang={Spanish}}
\begin{document}

\maketitle
\begin{frame}{Outline}
\tableofcontents
\end{frame}



\section{Indicaciones}
\label{sec:orga2a8dde}
\begin{frame}[fragile,allowframebreaks]{Indicaciones}
 \begin{itemize}
\item Recuerde que si options: H:2, entonces: 
\begin{itemize}
\item \texttt{*} Declara el nombre de la Sección
\item \texttt{**} Declara el nombre de la diapositiva
\end{itemize}
\item Puede alterar la estructura de la diapositiva si lo considera
necesario
\item Para este tema consulte las siguientes fuentes:
\begin{itemize}
\item \textcite{stallings2006}, 7ma edición, 2006, Español, Capítulo 5
página 171 y Capítulo 6 página 197 \autocite{stallings2006}.
\item \textcite{stallings2022computer}, 11ava edición, 2022, English,
Capítulo 4 desde página 136, Capítulo 5 desde página 162 y Capitulo
6 desde 201 \autocite{stallings2022computer}.
\end{itemize}
\item La tupla (E1,7, 97) significa Grupo E1, Libro Edición 7, página 97
del PDF (no del libro)
\item Las personas que exponen suben la presentación en formato .ORG y
.PDF
\item Las personas que atienden suben los apuntes de la clase de acuerdo al
formato de toma de notas en .ORG y .PDF
\end{itemize}
\end{frame}
\begin{frame}[label={sec:org1d8aa3f}]{Diseño de las Diapositivas}
\begin{itemize}
\item Para diseñar sus diapositivas puede consultar cualquiera de las
presentaciones .ORG desarrolladas por el profesor así como al
archivo \href{https://github.com/LeninGF/EPN-Lectures/blob/main/iccd332ArqComp-2024-B/Tutoriales/Beamer-Emacs/tutorialBeamer.org}{tutorialBeamer.org} en el repositorio de \href{https://github.com/LeninGF/EPN-Lectures/blob/main/iccd332ArqComp-2024-B/Tutoriales/Beamer-Emacs/tutorialBeamer.org}{GitHub} de la clase.
\item Recuerde que los archivos .ORG son archivos de texto así que los
puede copiar y sustituir por su texto propio.
\end{itemize}
\end{frame}
\begin{frame}[label={sec:orgb70bab0}]{Sobre este Documento}
\begin{itemize}
\item Este documento tiene la propuesta de temas a tratar y desarrollar
por los estudiantes.
\item Se ha de utilizar como base la bibliografía recomendada, pero puede
consultar bibliografía adicional.
\end{itemize}
\end{frame}
\section{Estructura del sistema de Memoria(E1, 11, 136)}
\label{sec:orgbbcc6a9}
\begin{frame}[label={sec:orgb28f405}]{Principio de Localidad (E1, 11, 137)}
\end{frame}
\begin{frame}[label={sec:org2d77c82}]{Características de los sistemas de Memoria (E1, 11, 142)}
\end{frame}
\begin{frame}[label={sec:orgc9669ff}]{Ubicación (E1)}
\end{frame}
\begin{frame}[label={sec:org952d763}]{Capacidad (E1)}
\end{frame}
\begin{frame}[label={sec:orgf48360d}]{Unidad de Transferencia (E1)}
\end{frame}
\begin{frame}[label={sec:orgf1a6e8d}]{Acceso secuencial (E1)}
\end{frame}
\begin{frame}[label={sec:orgad7f9b0}]{Acceso directo (E1)}
\end{frame}
\begin{frame}[label={sec:orge80f3cf}]{Acceso aleatorio (E1)}
\end{frame}
\begin{frame}[label={sec:orgd581a35}]{Tiempo de Acceso (E1)}
\end{frame}
\begin{frame}[label={sec:orgd69d8e0}]{Tiempo de ciclo de memoria (E1)}
\end{frame}
\begin{frame}[label={sec:org6b31941}]{Jerarquía de Memoria (E1, 11,145) hasta (E1,11,150)}
\end{frame}
\section{Memoria Cache (E2, 11, 162)}
\label{sec:org8bc9988}
\begin{frame}[label={sec:org78fd263}]{Principios Básicos de las Memorias Caché (E2,11,163)(E2,7,133)}
\begin{itemize}
\item Realice un resumen de lo más esencial del tema
\end{itemize}
\end{frame}
\begin{frame}[label={sec:org14539d4}]{Elementos de Diseño de la memoria Caché}
\end{frame}
\begin{frame}[label={sec:org46e052c}]{Tamaño Caché}
\end{frame}
\begin{frame}[label={sec:org4398bc6}]{Función de Correspondencia (E2,11,170)(E2,7,137)}
\begin{itemize}
\item Se recomienda la tabla 5.3 página 170 de la 10ma edición
\end{itemize}
\end{frame}
\begin{frame}[label={sec:org18bec8d}]{Algoritmo de Sustitución (E2,7,148)}
\end{frame}
\begin{frame}[label={sec:org3db2b60}]{Política de escritura}
\end{frame}
\begin{frame}[label={sec:org67c956b}]{Tamaño de Línea}
\end{frame}
\begin{frame}[label={sec:orgca54c87}]{Número de Cachés (E2, 7, 150)}
\end{frame}
\section{Memoria Interna (E3,7,172)(E3,11,201)}
\label{sec:org168de22}
\begin{frame}[label={sec:orga7882f7}]{Organización Memoria Principal Semiconductora (E3,7,172) (E3,11,201)}
\begin{itemize}
\item \autocite{stallings2006} página 172
\item \autocite{stallings2022computer} página 201 Capítulo 6
\end{itemize}
\end{frame}
\begin{frame}[label={sec:org0f75759}]{DRAM y SRAM}
\end{frame}
\begin{frame}[label={sec:orga42a711}]{RAM dinámica}
\end{frame}
\begin{frame}[label={sec:org8c7bcee}]{SRAM RAM estática}
\end{frame}
\begin{frame}[label={sec:orgec5a4db}]{Tipos de ROM}
\end{frame}
\begin{frame}[label={sec:org876cf24}]{Chip de Memoria RAM}
\end{frame}
\section{Corrección de Errores (E4, 7, 181)(E4,11,211)}
\label{sec:org9cbfe42}
\begin{frame}[label={sec:orgcb28f77}]{Hard Error vs Soft Error}
\end{frame}
\begin{frame}[label={sec:org0545cde}]{Código de Hamming}
\begin{itemize}
\item Realice una explicación sencilla con ejemplo
\end{itemize}
\end{frame}
\section{Organización Avanzada de Memorias RAM (E5, 7, 187)(E5,11,216)}
\label{sec:org55d01b7}
\begin{frame}[label={sec:org8c229ba}]{Dram síncrona}
\end{frame}
\begin{frame}[label={sec:orgaee377f}]{DDR SDRAM}
\end{frame}
\begin{frame}[label={sec:org9db7df4}]{EDRAM}
\end{frame}
\begin{frame}[label={sec:org7c59c9e}]{Flash Memory(E5,11,223)}
\end{frame}
\section{Memorias no volátiles de estado solido(E5,11,226)}
\label{sec:orgbccf795}
\begin{frame}[label={sec:org8060d2a}]{STT-RAM}
\end{frame}
\begin{frame}[label={sec:org33ce9f3}]{PCRAM}
\end{frame}

\section{Referencias}
\label{sec:orgc7f85ac}
\begin{frame}[allowframebreaks]{Bibliografía}
\printbibliography
\end{frame}
\end{document}
